\documentclass{article}
\usepackage[utf8]{inputenc}
\usepackage[T1]{fontenc}
\usepackage{amsmath,mathpazo}
\usepackage{parskip}

\title{SGSN word embedder optimization}
\author{Lazar Jelić}
\date{\today}

\begin{document}

\maketitle

\begin{abstract}

This note provides new insight on time and memory optimization techniques for
skip-gram word embedding neural network with autoencoder architecture.

\end{abstract}

\section{Introduction}

There are some ways to optimize computations and memory usage in the
original word2vec implementation. In this note we explore some of them.

\section{Implementation}

We are implementing word embedder using C programming language.

\subsection{Network architecture}

As in the original word2vec implementation, we are using autoencoder
architecture which is capable of discovering structure within data in order
to develop a compressed representation of the input. If we pass word from
vocabulary to the input layer, then this representation will be word embedding.

\medbreak

Network consists of $3$ layers and $2$ weight matrices between them.

The first layer is the input layer which is onehot vector that uniquely represents a given word in the vocabulary.
We can denote it's vector as $\boldsymbol{i}$ and it's size as $I$.

\medbreak

The second layer is the hidden layer which input is the ouput from the input
layer.
We can denote it's vector as $\boldsymbol{h}$ and it's size as $J$.

\medbreak

The third layer is the output layer which input is the ouput from the hidden
layer.
We can denote it's vector as $\boldsymbol{o}$ and it's size as $K$.

\medbreak

Weights between input and hidden layer. We can denote this matrix as $V$.

\medbreak

Weights between hidden and output layer. We can denote this matrix as $W$.

\subsection{Backpropagation}

Backpropagation is done using relationships defined by

\begin{align*}
	&p(w_{c,j} = w_{O,c} | w_I) = y_{c,j} = \frac{e^{u_{c,j}}}{\sum_{j^\prime=1}^V e^u_{j^\prime}} \\
	&u_{c,j} = u_j = \boldsymbol{v^{\prime\top}} \cdot \boldsymbol{h}
\end{align*}

\end{document}
